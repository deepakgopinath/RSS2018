\documentclass[conference]{IEEEtran}
\usepackage{times}
% numbers option provides compact numerical references in the text. 
\usepackage[numbers]{natbib}
\usepackage{multicol}
\usepackage[bookmarks=true]{hyperref}
\usepackage{graphics} % for pdf, bitmapped graphics files
\usepackage{graphicx}
\usepackage[numbers]{natbib}
\usepackage{multicol}
\usepackage[bookmarks=true]{hyperref}
\usepackage{amsmath,amssymb,latexsym,float,epsfig,subfigure}
\usepackage{amsmath} % assumes amsmath package installed
\usepackage{amssymb}  % assumes amsmath package installed
\usepackage{lipsum}
\usepackage[export]{adjustbox}
\usepackage[normalem]{ulem} % underline
\usepackage{wrapfig}
\usepackage{multirow}
\usepackage{balance}
\usepackage{color}
\usepackage{url}
\usepackage{booktabs}
\usepackage{pifont}
\newcommand{\argmax}{\arg\!\max}
\newcommand{\norm}[1]{\left\lVert#1\right\rVert}
\pdfinfo{
	/Author (Deepak Gopinath, Brenna D. Argall)
	/Title  (Mode Switch Assistance to Human Intent Disambiguation)
	/CreationDate (January 31 2017)
	/Subject (Robots)
	/Keywords (Robots)
}
% numbers option provides compact numerical references in the text. 


\pdfinfo{
   /Author (Homer Simpson)
   /Title  (Robots: Our new overlords)
   /CreationDate (D:20101201120000)
   /Subject (Robots)
   /Keywords (Robots;Overlords)
}

\begin{document}

% paper title
\title{Fisher Information based Disambiguation of Human Intent - An Experimental Evaluation}
\author{Author Names Omitted for Anonymous Review. Paper-ID [\textbf{163}]}
% You will get a Paper-ID when submitting a pdf file to the conference system
%\author{Deepak Gopinath and Brenna D. Argall}

%\author{\authorblockN{Deepak Gopinath}
%\authorblockA{Department of Mechanical\\Engineering,
%Northwestern University\\
%Evanston, Illinois 30332\\
%Email: deepakedakkattilgopinath2015\\@u.northwestern.edu}
%\and
%\authorblockN{Brenna D. Argall}
%\authorblockA{Department of Mechanical\\Engineering,
%		Northwestern University\\
%		Evanston, Illinois 30332\\
%		Email: brenna.argall@northwestern.edu}
%	}

%\author{Deepak Gopinath$^{1}$ and Brenna D. Argall$^{2}$
%	\thanks{Manuscript received: March, 1, 2016; Revised June,
%		7, 2016; Accepted June, 29, 2016.}
%}
% avoiding spaces at the end of the author lines is not a problem with
% conference papers because we don't use \thanks or \IEEEmembership


% for over three affiliations, or if they all won't fit within the width
% of the page, use this alternative format:
% 
%\author{\authorblockN{Deepak E. Gopinath\authorrefmark{1}\authorrefmark{2},
%		Brenna D. Argall\authorrefmark{1}\authorrefmark{2}\authorrefmark{3}\authorrefmark{4}
%	}
%	\authorblockA{
%		\authorrefmark{1}Department of Mechanical Engineering, Northwestern University, Evanston, IL}
%	
%	\authorblockA{\authorrefmark{2}Rehabilitation Institute of Chicago, Chicago, IL}
%	
%	\authorblockA{\authorrefmark{3}Department of Physical Medicine and Rehabilitation, Northwestern University, Chicago, IL}
%	
%	\authorblockA{\authorrefmark{4}Department of Electrical Engineering and Computer Science, Northwestern University, Evanston, IL}
%	
%	\authorblockA{{\tt\small deepakgopinath@u.northwestern.edu}}
%	\authorblockA{{\tt\small brenna.argall@northwestern.edu}}
%}

\maketitle

\begin{abstract}
In this paper, we treat the problem of intent disambiguation in terms of information theoretic principles. The effectiveness of assistive robots is closely related to their ability to infer the users' needs and intentions and provide appropriate types of assistance quickly, confidently and accurately.  We propose two different control mode selection schemes for intent disambiguation using notions of entropy and Fisher information in the context of assistive robotic manipulation. The proposed algorithms characterize the disambiguation capabilities of different control modes by the forward projection of probability distributions over goals and computing the relevant information theoretic measures. Our previous exploratory pilot studies revealed that the success of the disambiguation system depends on a variety of factors and choice of parameters. In order to thoroughly investigate the impact of the choice of various parameters and components, we present extensive simulation results both for point-robots that reside in different spaces as well as on the simulated robotic arm. Our results indicate that with the disambiguation algorithm the robot is able to assist earlier during task execution. Furthermore, the goal inferences are more accurate when operating in the disambiguating modes. 
\end{abstract}

\IEEEpeerreviewmaketitle

\section{Introduction}
Assistive shared-control machines such as robotic arms and myoelectric prosthesis have the potential to transform the lives of millions of people with motor impairments as a result of spinal cord injuries or brain traumas. These machines can promote independence, enhance the quality of lives and revolutionize the way they interact with society. These machines help to extend the mobility and manipulation capabilities of individuals thereby helping them perform activities of daily living in a more effective manner. 

The efficacy of assistive machines rely on their ability to infer the users' needs and intentions and is often a bottleneck for providing appropriate assistance quickly, confidently and accurately. Due to the dimensionality mismatch between the high-dimensional robots and the low-dimensional interfaces, the users' are typically constrained to produce control commands that do not reveal the intent more expressively. Therefore, intent inference is a necessary and crucial component to ensure appropriate assistance. Specifically, in assistive robotic manipulation, often the first step of a task is to reach for and grasp discrete objects in the environment. Therefore in this context, intent inference can be cast as a problem of maintaining the probability distribution over all possible discrete goals (objects) in the workspace. 

Our insight is that the information contained in control command issued in certain control modes is more useful for accurate inference and thereby will help the robot in providing the appropriate kind of assistance. In this work, we rely on information theoretic principles to quantify the information regarding intent disambiguation contained in different control modes and select control modes that maximize intent disambiguation.

\section{Related Work}

\section{Mathematical Formalism}
\subsection{Set Theoretic Treatment of Control Modes}
\subsection{Disambiguation Schemes}
\subsubsection{Fisher Information Based Metric}
\subsubsection{Entropy-based Metric}
\subsubsection{Heuristic-based Metric}

\subsection{Intent inference}
\subsubsection{Dynamic Field Theory}
\subsubsection{Bayesian Inference}
\subsubsection{Confidence Functions}

\subsection{Shared Control}

\section{Experimental Evaluation}

\section{Results}
\section{Discussion}
\section{Conclusion}
\section{References}



\section*{Acknowledgments}

%% Use plainnat to work nicely with natbib. 

\bibliographystyle{plainnat}
\bibliography{references}

\end{document}


