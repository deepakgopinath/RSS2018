\documentclass[conference]{IEEEtran}
\usepackage{times}
% numbers option provides compact numerical references in the text. 
\usepackage[numbers]{natbib}
\usepackage{multicol}
\usepackage[bookmarks=true]{hyperref}
\usepackage{graphics} % for pdf, bitmapped graphics files
\usepackage{graphicx}
\usepackage[numbers]{natbib}
\usepackage{multicol}
\usepackage[bookmarks=true]{hyperref}
\usepackage{amsmath,amssymb,latexsym,float,epsfig,subfigure}
\usepackage{amsmath} % assumes amsmath package installed
\usepackage{amssymb}  % assumes amsmath package installed
\usepackage{lipsum}
\usepackage[export]{adjustbox}
\usepackage[normalem]{ulem} % underline
\usepackage{wrapfig}
\usepackage{multirow}
\usepackage{balance}
\usepackage{color}
\usepackage{url}
\usepackage{booktabs}
\usepackage{pifont}
\newcommand{\argmax}{\arg\!\max}
\newcommand{\norm}[1]{\left\lVert#1\right\rVert}
\pdfinfo{
	/Author (Deepak Gopinath, Brenna D. Argall)
	/Title  (Mode Switch Assistance to Human Intent Disambiguation)
	/CreationDate (January 31 2017)
	/Subject (Robots)
	/Keywords (Robots)
}
% numbers option provides compact numerical references in the text. 


\pdfinfo{
   /Author (Homer Simpson)
   /Title  (Robots: Our new overlords)
   /CreationDate (D:20101201120000)
   /Subject (Robots)
   /Keywords (Robots;Overlords)
}

\begin{document}

% paper title
\title{Fisher Information based Disambiguation of Human Intent - An Experimental Evaluation}
\author{Author Names Omitted for Anonymous Review. Paper-ID [\textbf{163}]}
% You will get a Paper-ID when submitting a pdf file to the conference system
%\author{Deepak Gopinath and Brenna D. Argall}

%\author{\authorblockN{Deepak Gopinath}
%\authorblockA{Department of Mechanical\\Engineering,
%Northwestern University\\
%Evanston, Illinois 30332\\
%Email: deepakedakkattilgopinath2015\\@u.northwestern.edu}
%\and
%\authorblockN{Brenna D. Argall}
%\authorblockA{Department of Mechanical\\Engineering,
%		Northwestern University\\
%		Evanston, Illinois 30332\\
%		Email: brenna.argall@northwestern.edu}
%	}

%\author{Deepak Gopinath$^{1}$ and Brenna D. Argall$^{2}$
%	\thanks{Manuscript received: March, 1, 2016; Revised June,
%		7, 2016; Accepted June, 29, 2016.}
%}
% avoiding spaces at the end of the author lines is not a problem with
% conference papers because we don't use \thanks or \IEEEmembership


% for over three affiliations, or if they all won't fit within the width
% of the page, use this alternative format:
% 
%\author{\authorblockN{Deepak E. Gopinath\authorrefmark{1}\authorrefmark{2},
%		Brenna D. Argall\authorrefmark{1}\authorrefmark{2}\authorrefmark{3}\authorrefmark{4}
%	}
%	\authorblockA{
%		\authorrefmark{1}Department of Mechanical Engineering, Northwestern University, Evanston, IL}
%	
%	\authorblockA{\authorrefmark{2}Rehabilitation Institute of Chicago, Chicago, IL}
%	
%	\authorblockA{\authorrefmark{3}Department of Physical Medicine and Rehabilitation, Northwestern University, Chicago, IL}
%	
%	\authorblockA{\authorrefmark{4}Department of Electrical Engineering and Computer Science, Northwestern University, Evanston, IL}
%	
%	\authorblockA{{\tt\small deepakgopinath@u.northwestern.edu}}
%	\authorblockA{{\tt\small brenna.argall@northwestern.edu}}
%}

\maketitle

\begin{abstract}
The abstract goes here.
\end{abstract}

\IEEEpeerreviewmaketitle

\section{Introduction}
This demo file is intended to serve as a ``starter file" for the
Robotics: Science and Systems conference papers produced under \LaTeX\
using IEEEtran.cls version 1.7a and later.  

\section{Related Work}

Section text here. 

\section{Mathematical Formalism}
\subsection{Disambiguation Schemes}
\subsubsection{Fisher Information Based Metric}
\subsubsection{Entropy-based Metric}
\subsubsection{Heuristic-based Metric}

\subsection{Intent inference}
\subsubsection{Dynamic Field Theory Based}
\subsubsection{Bayesian Inference}
\subsubsection{Confidence Functions}

\subsection{Shared Control}

\section{Experimental Evaluation}


\section{RSS citations}

Please make sure to include \verb!natbib.sty! and to use the
\verb!plainnat.bst! bibliography style. \verb!natbib! provides additional
citation commands, most usefully \verb!\citet!. For example, rather than the
awkward construction 

{\small
\begin{verbatim}
\cite{kalman1960new} demonstrated...
\end{verbatim}
}

\noindent
rendered as ``\cite{kalman1960new} demonstrated...,''
or the
inconvenient 

{\small
\begin{verbatim}
Kalman \cite{kalman1960new} 
demonstrated...
\end{verbatim}
}

\noindent
rendered as 
``Kalman \cite{kalman1960new} demonstrated...'', 
one can
write 

{\small
\begin{verbatim}
\citet{kalman1960new} demonstrated... 
\end{verbatim}
}
\noindent
which renders as ``\citet{kalman1960new} demonstrated...'' and is 
both easy to write and much easier to read.
  
\subsection{RSS Hyperlinks}

This year, we would like to use the ability of PDF viewers to interpret
hyperlinks, specifically to allow each reference in the bibliography to be a
link to an online version of the reference. 
As an example, if you were to cite ``Passive Dynamic Walking''
\cite{McGeer01041990}, the entry in the bibtex would read:

{\small
\begin{verbatim}
@article{McGeer01041990,
  author = {McGeer, Tad}, 
  title = {\href{http://ijr.sagepub.com/content/9/2/62.abstract}{Passive Dynamic Walking}}, 
  volume = {9}, 
  number = {2}, 
  pages = {62-82}, 
  year = {1990}, 
  doi = {10.1177/027836499000900206}, 
  URL = {http://ijr.sagepub.com/content/9/2/62.abstract}, 
  eprint = {http://ijr.sagepub.com/content/9/2/62.full.pdf+html}, 
  journal = {The International Journal of Robotics Research}
}
\end{verbatim}
}
\noindent
and the entry in the compiled PDF would look like:

\def\tmplabel#1{[#1]}

\begin{enumerate}
\item[\tmplabel{1}] Tad McGeer. \href{http://ijr.sagepub.com/content/9/2/62.abstract}{Passive Dynamic
Walking}. {\em The International Journal of Robotics Research}, 9(2):62--82,
1990.
\end{enumerate}
%
where the title of the article is a link that takes you to the article on IJRR's website. 


Linking cited articles will not always be possible, especially for
older articles. There are also often several versions of papers
online: authors are free to decide what to use as the link destination
yet we strongly encourage to link to archival or publisher sites
(such as IEEE Xplore or Sage Journals).  We encourage all authors to use this feature to
the extent possible.

\section{Conclusion} 
\label{sec:conclusion}

The conclusion goes here.

\section*{Acknowledgments}

%% Use plainnat to work nicely with natbib. 

\bibliographystyle{plainnat}
\bibliography{references}

\end{document}


