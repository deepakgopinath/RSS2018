% Copyright Javier Sánchez-Monedero.
% Please report bugs and suggestions to (jsanchezm at uco.es)
%
% This document is released under a Creative Commons Licence 
% CC-BY-SA (http://creativecommons.org/licenses/by-sa/3.0/) 
%
% BASIC INSTRUCTIONS: 
% 1. Load and set up proper language packages
% 2. Complete the paper data commands
% 3. Use commands \rcomment and \newtext as shown in the example

\documentclass[a4paper,twoside,11pt]{reviewresponse}

% 1. Load and set up proper language packages
%\usepackage[utf8x]{inputenc}
\usepackage[latin9]{inputenc}
\usepackage[T1]{fontenc}
\usepackage[english]{babel}
\usepackage{amsmath}
\usepackage{amssymb}  % assumes amsmath package installed

% 2. Complete the paper data
\newcommand{\myAuthors}{{Deepak E. Gopinath$^{\displaystyle 1, 3}$, ~Brenna D. Argall$^{\displaystyle 1,2,3}$, }}
\newcommand{\myAuthorsShort}{John.~Doe et. al}
\newcommand{\myEmail}{deepak.gopinath@u.northwestern.edu}
\newcommand{\myTitle}{Information-Theoretic Characterization of Control Modes for Intent Disambiguation - Response to Reviewers}
\newcommand{\myShortTitle}{Response to reviewers}
\newcommand{\myJournal}{IEEE Robotics And Automation Letters (RA-L)}
\newcommand{\myDept}{{$^{\displaystyle 1}$Department of Mechanical Engineering, Northwestern University, Evanston, IL}\\
{$^{\displaystyle 2}$ Departments of Computer Science and Physical Medicine and Rehabilitation, Northwestern University, Evanston, IL  }\\
{$^{\displaystyle 3}$ Shirley Ryan AbilityLab, Chicago, IL }\\}
%%%%%%%%%%%%%%%%%%%%%%%%%%%%%%%%%%%%%%%%%%%%%%%%%%%%%%%%%%%%%%%%%%%%%%%%%%


%\usepackage[linktoc=all]{hyperref}
\usepackage[linktoc=all,bookmarks,bookmarksopen=true,bookmarksnumbered=true]{hyperref}

\hypersetup{
pdfauthor = {\myAuthorsShort},
pdftitle = {\myTitle},
pdfsubject = {\myJournal\xspace},
colorlinks = true,
linkcolor=black!70!green,          % color of internal links
citecolor=black!70!green,        % color of links to bibliography
filecolor=magenta,      % color of file links
urlcolor=black!70!green           % color of external links
}

\begin{document}

\thispagestyle{plain}

\begin{center}
 {\LARGE\myTitle} \vspace{1cm} \\
 {\large\myJournal} \vspace{0.5cm} \\
 \today \vspace{0.5cm} \\
 \myAuthors \\
 \url{\myEmail} \vspace{1cm} \\
 \myDept
\end{center}

%\tableofcontents

%\begin{abstract}
We sincerely thank the reviewers for their valuable efforts and feedback. We have made every effort to address all the major points raised by the reviewers and have modified the manuscript accordingly. For added ease of reading, we are also submitting a version of the revised manuscript in which the modifications are highlighted in orange color. Detailed responses to the comments follow.  
%\end{abstract}

\section{Reviewer 3}

\rcomment{- I think it is better to define certain terms such as "belief over
	goals" and "intention disambiguation/inference" in the introduction.}

We thank the reviewer for this comment and have modified the text to make the meaning of intent disambiguation and inference more clear to the reader.

\rcomment{As I understand Algorithm 1 explains how intentions are obtained,however it does not explain how this intentions are embedded in the
	robot assistive controller. }

If we are correctly interpreting the reviewer's comment, we believe 
there is a point of misinterpretation here: Algorithm 1 in fact is about 
computing the disambiguation mode, which is different from predicting 
the intent itself (if that is what the reviewer meant by "how intentions 
are obtained"). This distinction is indeed a potential point of 
confusion, and we have included a new figure (Figure 1) that explicitly 
calls out these various components as well as clarifying text throughout 
the document.

%
%We apologize the confusion that arose to the reviewer, but we would like to point that computing the disambiguation mode is different from predicting the intent itself. Algorithm 1 is about characterizing intent disambiguation capabilities of each control mode. 
%
%We would like to emphasize that the robot assistance, intent inference and intent disambiguation all are working in tandem as the human (simulated) interacts with the robot. In order to bring this point across, we have added a new figure which depicts how the different components interact with each other. We sincerely hope that this will clarify how the output of Algorithm 1 is integrated within the robot assistance system. 

\rcomment{Simulation methods are not described with sufficient details. It is
	hard to reproduce this. Could you add equations for human and robot
	control behaviour ($u_h$ and $u_r$)? Adding detailed equations and
	parameters to Appendix and supplementary material will be helpful.}

Tables I and II now provide all simulation parameters and their ranges. 
The specification of $u_h$ is given in Equation 7, and $u_r$ in Section 
V.A.1. (RA-L does not allow Appendix and supplementary material, we 
therefore were unable to submit those.)

%We have made every attempt to make the simulation details clear. Tables I and II provides all the simulation parameters and their ranges. We had already specified how $u_h$ (Equation 7) and $u_r$ were generated (Section V.A.1). We have modified Section V.B.3 to back reference Equation 7 and V.A.1. RA-L does not allow Appendix and supplementary material and therefore we won't be able to submit those. 

\rcomment{
	- I am not sure if it is valid to do statistical analysis of simulated
	data? }

We would like to respectfully argue that statistical analyses of 
simulated data is indeed valid (and happens often within the field of 
robotics). The goal with our simulation was to be able to control for 
various confounding factors that can potentially arise in real human subjects studies (such as fatigue, lack of attention etc.) and perform a more 
extensive evaluation (8000 trials for each metric space) than could ever be possible on real 
hardware and with real subjects. We do not believe that this replaces hardware evaluation, but 
rather that it complements it. In this work we chose to do the more 
exhaustive simulation evaluation first.

%
%We would like to respectfully disagree with the reviewer on this point. The goal with our simulation is to be able to control for various confounding factors and be able to randomize them properly so that the analysis can focus on the factor of interest, which in our case is efficacy of our intent disambiguation algorithms. Simulated data serve as a proxy/an approximation of what real data might look like. for example, this is similar to how modern deep learning techniques use data augmentation methods to create `synthetic data' to train their systems which can then be deployed in real scenarios. Training on simulated data does not invalidate the models learned but only improves their performance in the real world. 
\clearpage

\section{Reviewer 4}

\rcomment{One unclear issue is the presence of Dynamic Neural Field (DNF)
	approaches. Because the KL and Entropy algorithm utilizes the intent
	inference algorithms of heuristic, Bayesian, and dynamic neural field
	approaches. The other two algorithms were applied in this paper, but
	the usage of the DNS approach is not explicitly written.}

We apologize for this confusion. For the simulations we indeed utilized all three approaches (heuristic, Bayesian and DNF). Table I, Row 3 indicates the different intent inference approaches utilized in the simulations. 

\rcomment{, most of terms are salient but I suggest changing
	one of those terms to improve the readability. }

We thank the reviewer for this valuable suggestion and have modified the terms used for describing the different metrics accordingly for improved readability throughout the paper as well as in the figures. `Total Amount of Assistance' has been changed to `Temporal Ratio of Assistance'. 

\rcomment{ unclear selection of test
	condition in the second experiment... KL and ENT should be compared to
	GRD or ENT should be compared to GRD at least because it showed better
	performance...}

We agree with the reviewer that a full comprehensive experiment would need to compare both ENT and KLD against GRD condition. As the reviewer rightly spotted, this paper is an extension to our earlier paper [21] which just relied on hand-crafted heuristic features to perform intent disambiguation. 
Our goal was to identify at least one 
information-theoretic approach that was better than the heuristic 
approach (if one existed). That KLD-based intent disambiguation system 
was found to be more effective than the GRD condition validated the use 
of information theoretic approaches for intent disambiguation. Having 
validated the promise of the information-theorectic approaches, we are 
now in the process of preparing a full-fledged human subject study in 
which one of the conditions indeed will be to test both ENT and KLD 
against the GRD.

%Our goal was to identify at least one information-theoretic approach than could possibly be better than the heuristic approach. We are in the process of extending the current study into a full fledged human subject study in which we will be testing both ENT and KLD against the GRD. We were however, unable to finish the experiment in time for the submission deadline. 

%We would still like to point out that the KLD-based intent disambiguation system was still more effective than the GRD condition thereby marking an improvement over a heuristic based approach and validates the use of information theoretic approaches for intent disambiguation. 

%We would also like to emphasize that the major difference between the point robot simulations and the robotic arm simulation setup is the nonlinear kinematics. The robot autonomy was still generated using the potential field approach. 

\clearpage
\section{Reviewer 9}

\rcomment{It seems that the approach is specialised for a specific task, i.e. for
	a reaching task where goals are discrete. The authors claim that this
	type of task is the most common which seems to be an overstatement.}

We sincerely believe that this is not an overstatement, given the 
capabilities of today's assistive robotic arms. In the domain of assistive robotic manipulations,  robotic arms are typically used by motor-impaired subjects for performance activities of daily living (ADL) tasks. Various types of ADL tasks such feeding, picking up objects from shelves, tabletops and the ground, reaching for doorknobs all involve reaching toward objects in the world. We have added text to the introduction where we clarify this point. Numerous videos of use-cases (available online) in which motor impaired people use robotic arms in their daily lives will further illustrate our point. 

\rcomment{There are many tasks in robot assistance or HRI that may benefit from
	assistance, e.g. obstacle avoidance, trajectory tracking, force
	tracking, etc. The authors are encouraged to discuss how does the
	proposed approach generalise to these tasks.
}

We thank the reviewer for this suggestion and accordingly we have added some text in the discussion section addressing how generalizable our algorithm is. We make further reference to the generalizability in the introduction section where we state that intent inference is crucial for the success for many shared-autonomy human machine systems. Any human-robot system that has an intent inference mechanism under-the-hood can potentially utilize our intent disambiguation algorithm with respect to relevant parameters that affect intent inference (control modes, in our specific example) to enhance the efficacy of the inference mechanism.
\rcomment{
	Related work rather lists some scarce examples instead of being a
	structured summation of the most important works related to this paper.
	There are many approaches for intent inference found in the literature,
	see the works of e.g. Anca Dragan. Also, the related work should be
	linked to the present work in a meaningful way. }

We thank the reviewer for this suggestion and have accordingly made our related work section more comprehensive. We hope this modification will help the reader to make the connection between past work and the work presented in the paper more clear as well.

\rcomment{It is unnecessary to devote the whole section to mathematical notation.
	It should only be a paragraph in the introduction. It is highly
	advisable to rather give a problem setting, with accompanying figure of
	the task considered where the notation is usefully presented}

We have taken this suggestion into account and have accordingly deleted the section devoted to mathematical notation. The content has been integrated with the new Section IV. 
\rcomment{Please list all the assumptions you impose on the problem in one place,
	e.g. that the human doesn't change the goal, that at any given time
	only one control mode can be activated, etc. Mentioning assumptions "on
	the fly" is not suitable.}

We thank the reviewer for the comment, but we have respectfully chosen not to change the way we present the math. We believe that it is important to start with the most general form of 
the equations and to introduce simplifications based on different 
assumptions as the problem specification becomes less general. In our 
experience such an approach helps the reader to better follow the 
progression of ideas and makes clear the simplifications. 

\rcomment{Can you explain what would happen with the proposed approach if there
	is an obstacle to a goal?}

This is an important observation by the reviewer. An obstacle to a goal can indeed cause 'unintuitive changes' to the direct path to a goal. However, we would like to emphasize that, how well the autonomy is able to reason about these kinds of changes in motion direction depends on the features (directedness, proximity, agreement etc) used by the \textbf{intent inference} module. That ability of the intent disambiguation system to disambiguate intent is only as effective as the underlying intent inference mechanism. This is because, during the forward projection of beliefs (Line 7 in Algorithm 1), the BeliefUpdate() utilizes the same inference algorithm that is active during task execution. If the choice of intent inference algorithm ignores the presence of obstacles (probably not the best choice!), intent disambiguation will not be able to reason about the presence of obstacles. 

One can think of the intent disambiguation as an add-on module to whichever intent inference scheme is used, such that the disambiguation system is able to rank order the (discrete) parameters on which the inference depends on according to their ability to extract information that will help to enhance inference accuracy.

%his is an interesting question and we would like to point out that the `intent disambiguation' algorithm is designed to improve the accuracy of the intent inference module. The robot autonomy is generated from a policy that is \textbf{independent} of the intent disambiguation algorithm. In our case, we utilize a potential field based approach to generate robot autonomy signals. The potential field generator utilizes the intent inference module's prediction of what the current goal and treats the predicted goal as the sole `attractor in the world'. All the other objects in the world are then treated as obstacles and therefore will have associated repeller fields. We have modified Section V.A.1 to include this information and we hope this would clarify this point. 

\rcomment{..Where are the results for the dynamic neural field approach?  A
	multitude of different approaches is proposed which is exceptionally
	confusing and it is very hard to follow what is relevant in this paper.
	Or to rephrase, how disambiguation, inference, and assistance concepts
	participate in the evaluation..?}

We have cited one of our other works which does a qualitative comparison of dynamic neural field approach to other standard approaches for intent inference.

The reason why we have used a multitude of approaches for each 
component of the overall system was to more rigorously evaluate the 
robustness of our proposed intent disambiguation approach---by 
demonstrating that its effectiveness was not contingent on a particular 
choice of intent inference algorithm, shared control algorithm or 
disambiguation metric. We hope that Figure 1 helps to clarify how all of 
the different components interact with each other.

% We would like to argue that the reason why we have used a multitude of approaches for each component of the overall system is that, in simulation we wanted to ensure the robustness of our proposed intent disambiguation approach to the choice of different algorithms for other components of the system. We would like to point out that the intent inference, intent disambiguation and shared control assistance are all working in tandem as the human interacts with the robot. This is exemplified in the Figure 1, that depicts how the different components work together. We hope this clarifies how all the different components `participate in the evaluation'. 

\rcomment{..The simulation environment, task, and the complete evaluation setup
	need to be thoroughly explained. Since the actual user study is
	missing, it is very important to describe in detail (with equations)
	how is the human simulated.}

\textbf{Response}

Thank you for the suggestion. We have described the simulation 
environment in greater detail, and hope that the modifications in the 
current version make the simulation setup more comprehensible to the 
reader. In particular, Section V.B.3 describes how the human actions are simulated. Further details of how the simulations were conducted are also described in Section V.B.5 and in simulation parameters and their ranges are presented were also presented in Tables I and II. Furthermore, we are also making the point robot simulation code open source and is now available on GitHub. The paper contains the link to the repo as well.


\end{document}
